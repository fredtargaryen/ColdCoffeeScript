\documentclass{article}

\usepackage[T1]{fontenc}
\usepackage[utf8]{inputenc}
\usepackage{lmodern}
\usepackage{listings}
	
\author{James Curran and Shahaan Hassan}
\title{ColdCoffeeScript User Guide}

\begin{document}
\maketitle
ColdCoffeeScript, its interpreter and accompanying features, and this User Guide are in no way affiliated with or endorsed by popular University of Southampton lecturer Gennaro Parlato.  
\tableofcontents % for a table of contents

\section{Introductory Quotes and Haiku}
\begin{quote}
\textit{"You can't put that lmao"}
\end{quote}

 - Shahaan Hassan, 2017

\begin{quote}
\textit{We have distilled the\\One and only Gennaro\\Into a language\\\\If you want some proof\\that our language is dank, well\\we got 4/20}
\end{quote}

\section{"Serious" Introduction}
ColdCoffeeScript is a strongly and statically typed imperative domain-specific language, created for the purpose of manipulating regular languages. The language prides orthogonality over syntactic sugar so the developers have DELIBERATELY AIMED for a minimal syntax. All functions and operators are pass-by-value - variables are only changed when explicitly re-assigned by the programmer.

\section{Features}
ColdCoffeeScript boasts many organic features, guaranteed to make programming simply delightful. These include:
\begin{itemize}
\item Delicious Gennaro-themed error messages, including such choice phrases as "" and "" (we only ever type valid programs so we don't know)
\item Incredible type checker to whip those unruly variables into shape before they go into the program!
\item Single line comments - include any old muck in your program by surrounding part of your line with hashes, like so \#Shahaan remember to add more features here\#
\end{itemize}
\section{Syntax}
So, essentially, the idea ees-a, I will give-a you a taybul of the syntax and what eet-a does. Then you can mess around with it for homework.

Nota Bene: Variable names are in capitals only. Thees-a was done deliberately to distinguish variables from functions, and emphasise the importance of good variable naming.
\subsection{Hello World}
Thees ees-a the anatomy of a Hello World Program in ColdCoffeeScript.
\begin{lstlisting}
display {"hello","world"} 2;
\end{lstlisting}
On line 1, 'display' ees-a the function used to output a set to standard output. The output can-a probably be redirected as-a well.

On line 1, '\{"hello","world"\}' ees an exampul of what ees known in ColdCoffeeScript as a set literal. The parentheses show the start and end of the set, and the set contains the two strings "hello" and "world", separated by a comma.

On line 1, '2' ees the number of elements of the set that will be printed. Thees ees-a useful for-a very large sets where you don't care about all of the output.

The semicolon ends the display statement. If you do not end each statement weeth a semicolon, your statement will be just a concept and your DFA will-a fail.

The output of the program is as follows:
\begin{lstlisting}
{ciao,mondo}
\end{lstlisting}
This is because the language automatically translates English strings to Italian.
\newpage
\subsection{Not really}
The actual output is:
\begin{lstlisting}
{hello,world}
\end{lstlisting}
\section{Frequently Asked Questions}
\end{document}

